\chapter{Software Engineering Process}
\section{Methodology}
\subsection{General Overview}
The project was carried out with use of Agile methodologies. Weekly supervisor meetings were held, in which the past week's work would be evaluated. This was then contextualised within the overall time frame of the project. This critical analysis helped to identify and set goals for the next week and beyond. Agile development suited the nature of the project as the full progression of the project was not clear from the start. For example, initially it was planned to apply a \acrlong{wfc} implementation directly and focus more on extending it to aid game design. However, while there were many implementations of \acrshort{wfc} available online, many of them had poor documentation and did not work out of the box due to missing assets and errors, while others could not be applied to a 3D tile set. The official \acrlong{wfc} GitHub contains links to other \acrshort{wfc} implementations \cite{Gumin_Wave_Function_Collapse_2016}. Four implementations investigated were the original implementation, two forks by Joseph Parker \cite{unity-WFC} and Maksim Priakhin \cite{unity-WFC-3D}, as well as a simplified implementation by Garnet Kane \cite{Easy_WFC}.

\subsection{Semester One}
% How meetings were held.
% What meetings consisted of.
% How what was discussed in meetings changed over time. I.e. earlier meetings consisted of... Second semester consisted of... This problem came up. I did this to solve it... (STAR)
The key areas of focus in the first semester were carrying out a literature review, setting objectives, reviewing ethics, designing the game and implementing the \acrshort{wfc} algorithm. As described, the goal of the project was initially to extend upon an existing implementation of \acrshort{wfc}. As this was unsuccessful, the focus changed to implementing an algorithm from scratch. After the core of the constraint solver was finished, \acrshort{wfc}'s lowest entropy cell selection and random weighted tile selection were added. The meeting notes for semester one are available in Appendix Section \ref{sec:semester_one_meeting_notes}.

\subsection{Semester Two}
The key areas of focus in the second semester were finishing the game and documenting the project in this report. The core generation had already been fully implemented, but assets still had to be created and put into the generator. Furthermore, extensions on the constraint solver, such as \acrlong{imib} and dynamic chunk loading, had to be added to make levels infinite and playable. Additional work during the holidays involved studying \acrlong{imib}, adding graphics filters and starting block modelling. A list of tasks was created during this time and extended during semester two. This list and the meeting notes for semester two are available in Appendix Sections \ref{sec:tasks} and \ref{sec:semester_two_meeting_notes} respectively.

\section{Tools and Technologies}
\subsection{Unity (C\#)}
Unity was used as the game engine for development. This was chosen as Unity and its scripting API language C\# have commonly been used for implementations of \acrlong{wfc}, including the original \acrshort{wfc} repository by Maxim Gumin \cite{Gumin_Wave_Function_Collapse_2016}. \acrshort{wfc} has also been adapted to other engines such as Unreal Engine \cite{unreal_engine_WFC}. Two issues facing later development using Unity were its poor support for multithreading and importing \texttt{.fbx} models from Blender.

\subsection{Blender}
Blender was used to create tile models as it is a free but powerful modelling software. Each model could be created in Blender and exported as an \texttt{.fbx} file. These files were imported into Unity and unpacked. Each model could then be used as a GameObject and have any additional assets such as lights and audio sources attached. Due to Unity's poor support for loading files from Blender, materials had to be reassigned in Unity.

\subsection{GitHub}
GitHub was used for version control. This was useful for comparing new and old code and tracking progress over time.

\subsection{Document Management}
Google Drive and Google Docs, both part of Google Workspace, were used to hold most documents relating to the project. This included a tasks document (Appendix Chapter \ref{sec:tasks}), game design document (Appendix Chapter \ref{sec:designDocument}), weekly meeting notes (Appendix Chapter \ref{sec:meeting_notes}), literature review research document and credits document for any external resources used. Furthermore, Overleaf was used to write up the majority of this dissertation. These cloud platforms enabled work from multiple locations and more effective evaluation with the project supervisor as the latest versions of documents were always available. One issue with Overleaf is that it has a compile time limit. The full version of the disseration would time out on Overleaf, so instead had to be edited and compiled locally. GitHub was again used to provide version control.