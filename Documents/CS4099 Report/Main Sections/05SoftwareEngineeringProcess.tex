\chapter{Software Engineering Process}
\section{Methodology}
\subsection{General Overview}
The project was carried out with use of Agile methodologies. Weekly supervisor meetings were held, in which the past week's work would be evaluated. This was then contextualised within the overall time frame of the project. This critical analysis helped to identify and set goals for the next week and beyond. Agile development suited the nature of the project as the full progression of the project was not clear from the start. For example, initially it was planned to apply a Wave Function Collapse implementation directly and focus more on extending it to aid game design. However, while there were many implementations of WFC available online, many of them had poor documentation and did not work out of the box due to missing assets and errors, while others could not be applied to a 3D tile set. The official Wave Function Collapse GitHub contains links to other WFC implementations \cite{Gumin_Wave_Function_Collapse_2016}. Four implementations investigate were the original implementation, two forks by Joseph Parker \cite{unity-WFC} and Maksim Priakhin \cite{unity-WFC-3D} as well as a simplified implementation by Garnet Kane \cite{Easy_WFC}.

\subsection{Semester One}
% How meetings were held.
% What meetings consisted of.
% How what was discussed in meetings changed over time. I.e. earlier meetings consisted of... Second semester consisted of... This problem came up. I did this to solve it... (STAR)
The key areas of focus in the first semester were carrying out a literature review, setting objectives, reviewing ethics, designing the game and implementing the WFC algorithm. As described, the goal of the project was initially to extend upon an existing implementation of WFC. As this was unsuccessful, the focus changed to implementing an algorithm from scratch. After the core of the constraint solver was finished, WFC's lowest entropy cell selection and random weighted tile selection were added. The meeting notes for semester one are available in appendix section \ref{sec:semester_one_meeting_notes}.

\subsection{Semester Two}
The key areas of focus in the second semester were finishing the game and documenting the project in this report. The core generation had already been fully implemented, but assets still had to be created and put into the generator. Furthermore, extensions on the constraint solver, such as infinite modifying in blocks and starting block constraints, had to be added to make levels infinite and playable. Additional work during the holidays involved studying infinite modifying in blocks, adding graphics filters and starting block modelling. A list of tasks was created during this time and extended during semester two. This list and the meeting notes for semester two are available in appendix sections \ref{sec:tasks} and \ref{sec:semester_two_meeting_notes} respectively.

\section{Tools and Technologies}
\subsection{Unity (C\#)}
Unity was used as the game engine for development. This was chosen as Unity and its scripting API language C\# have commonly been used for implementations of Wave Function Collapse, including the original WFC repository by Maxim Gumin \cite{Gumin_Wave_Function_Collapse_2016}. WFC has also been adapted to other engines such as Unreal Engine \cite{unreal_engine_WFC}. Two issues that faced later development using Unity were its poor support for multithreading and importing \texttt{.fbx} models from Blender.

\subsection{Blender}
Blender was used to create tile models. Each model could be created and textured before being exported as \texttt{.fbx} files. These were then imported into Unity and unpacked. They could then be used as GameObjects and have any additional assets such as lights and audio sources attached.

\subsection{GitHub}
GitHub was used for version control. This was useful for comparing new and old code and tracking progress over time.

\subsection{Document Management}
Google Drive and Google Docs, both part of Google Workspace, were used to hold most documents relating to the project. This included a tasks document, game design document, weekly meeting notes, literature review research document and credits document for any external resources used. Furthermore, Overleaf was used to write up this report. These online platforms enabled work from multiple locations and more effective evaluation with the project supervisor as the latest versions of documents were always available.