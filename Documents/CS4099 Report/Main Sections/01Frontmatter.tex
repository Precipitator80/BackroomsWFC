%----------
%	Abstract
%----------	
\section*{Abstract}
% Give an abstract of what was achieved.
% This section should contain a short overview of the practical: 
%   • What were you asked to do?
%   • What did you achieve?
%   • Which parts have you completed and to what extent?
% Exciting summary of introduction and conclusion.
% Motivation behind topic.
Wave Function Collapse (WFC) is a Procedural Content Generation (PCG) technique that enables generation of large output from limited input. From games \cite{townscaper,badnorth,cavesofqud} to poetry \cite{WFC_poetry} and music \cite{WFC_music,WFC_music_2}, WFC has seen application across a huge number of fields. However, many implementations available online suffer from issues like a lack of global constraints as well as poor performance and documentation.

% What were the goals and what did you achieve
This project contextualises WFC within theory of constraint programming and implements simple-tiled WFC using the Maintaining Arc Consistency 3 algorithm. WFC is extended to generate infinite worlds using Infinite Modifying in Blocks \cite{Infinite_Modifying_In_Blocks}. An interface is provided for the Unity Editor that allows designers to specify their own tile set to use for generation. As example, a tile set themed after the popular fictional concept of `the Backrooms' is created.

% How can work be extended - Move to evaluation / conclusion / future work.
%With additional time, the solver could be optimised to use multi-threading and lazy loading to improve performance. Furthermore, the Unity Editor interface could be polished to make it easier to specify tile sets. Finally, the Backrooms game demo could be extended through additional models, puzzles and levels.

%\newpage
\section*{Declaration}
% Usual stuff with word count and so on
I declare that the material submitted for assessment is my own work except where credit is explicitly given to others by citation or acknowledgement. This work was performed during the current academic year except where otherwise stated.
\par \vspace{\baselineskip}
The main text of this project report is XXXXX words
long, including project specification and plan.
\par \vspace{\baselineskip}
In submitting this project report to the University of St
Andrews, I give permission for it to be made available
for use in accordance with the regulations of the University
Library. I also give permission for the title and abstract
to be published and for copies of the report to be made and
supplied at cost to any bona fide library or research worker,
and to be made available on the World Wide Web. I retain the copyright in this work.
%\newpage

\section*{Acknowledgements}
I would like to thank all those who supported me throughout the time of the project, from university staff to family and friends.