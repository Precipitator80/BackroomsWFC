%----------
%	Abstract
%----------	
\section*{Abstract}
% Give an abstract of what was achieved.
% This section should contain a short overview of the practical: 
%   • What were you asked to do?
%   • What did you achieve?
%   • Which parts have you completed and to what extent?
% Exciting summary of introduction and conclusion.
% Motivation behind topic.
% Begin with the PROBLEM. Next sentence discuss 'existing literature'. Next sentence begin with what your diss addresses. Then how it goes about it. Then findings. Then limitations. % Overall
\acrfull{wfc} is limited by a lack of global constraints, poor performance and restriction to a finite grid. These issues are rarely addressed directly in implementations. Instead, they are worked around in an ad hoc, game-specific way that fails to exploit constraint programming techniques. Furthermore, presentation of \acrshort{wfc} online often fails to acknowledge underlying constraint solving principles used by \acrshort{wfc}. This dissertation seeks to examine how these issues can be addressed and attempts to contextualise \acrshort{wfc} within theory of constraint programming. To do this, a simple-tiled implementation of \acrshort{wfc} using the \acrfull{mac3} algorithm is presented. \acrshort{wfc} is extended to generate infinite worlds using \acrfull{imib}. An interface is provided for the Unity Editor that allows designers to specify their own tile set to use for generation. As example, a game themed after the popular fictional concept of \textit{the Backrooms} is created. Weighted tile selection gives the level designer global control over the percentage of each tile in the output level. However, loading infinite worlds in real-time presents additional performance challenges and further global constraints are needed to improve control further. Overall, this dissertation brings together \acrshort{wfc} and constraint programming concepts with extensive examples to support understanding and further development of \acrshort{wfc}.

% OLD ABSTRACT
%\acrfull{pcg} describes the use of algorithms to pseudo-randomly generate content. \acrfull{wfc} describes a family of \acrshort{pcg} algorithms that generate large output from limited input. From games \cite{townscaper,badnorth,cavesofqud} to poetry \cite{WFC_poetry} and music \cite{WFC_music,WFC_music_2}, \acrshort{wfc} has seen application across a huge number of fields. However, many implementations available online suffer from issues such as a lack of global constraints as well as poor performance and documentation.

% What were the goals and what did you achieve
%This project contextualises \acrshort{wfc} within theory of constraint programming and implements simple-tiled \acrshort{wfc} using the \acrlong{mac3} algorithm. \acrshort{wfc} is extended to generate infinite worlds using \acrfull{imib} \cite{Infinite_Modifying_In_Blocks}. An interface is provided for the Unity Editor that allows designers to specify their own tile set to use for generation. As example, a tile set themed after the popular fictional concept of \textit{the Backrooms} is created.

% How can work be extended - Move to evaluation / conclusion / future work.
%With additional time, the solver could be optimised to use multi-threading and lazy loading to improve performance. Furthermore, the Unity Editor interface could be polished to make it easier to specify tile sets. Finally, \textit{the Backrooms} game demo could be extended through additional models, puzzles and levels.

%\newpage
\section*{Declaration}
% Usual stuff with word count and so on
I declare that the material submitted for assessment is my own work except where credit is explicitly given to others by citation or acknowledgement. This work was performed during the current academic year except where otherwise stated.
\par \vspace{\baselineskip}
The main text of this project report is 10616 words long, including project specification and plan.
\par \vspace{\baselineskip}
In submitting this project report to the University of St Andrews, I give permission for it to be made available for use in accordance with the regulations of the University Library. I also give permission for the title and abstract to be published and for copies of the report to be made and supplied at cost to any bona fide library or research worker, and to be made available on the World Wide Web. I retain the copyright in this work.
%\newpage

\section*{Acknowledgements}
I would like to thank all those who supported me throughout the time of the project, from university staff to family and friends.