\appendix
\chapter{Testing Summary}
% Describe how you tested your program. Include any ways you made sure that your program is complete and works correctly.

\chapter{Player Guide}
\subsubsection{Premise}
% Explore an infinite, procedurally generated interpretation of the Backrooms! Try to find a way out, traversing a liminal space

Embark on an infinite journey through the enigmatic depths of the Backrooms. Take in the eerie solitude in this procedurally generated liminal space, where the boundaries between reality and the unknown blur into obscurity. Scavenge for any items left behind and piece together an escape at the elusive escape points scattered throughout. Will you manage to find an exit, or will you become lost forever in this maze of yellow hues and buzzing lights?

% Image of backrooms in-game.

% AI Text 1: Embark on an endless journey through the enigmatic depths of the Backrooms, where each corridor holds a new mystery waiting to be unraveled. In this procedurally generated world, solitude envelops you as you navigate through the eerie confines of this liminal space. Your objective: find the elusive escape points scattered throughout, piecing together clues and gathering essential items along the way. Will you brave the unknown, or succumb to the labyrinthine depths of the Backrooms? The choice is yours in this haunting adventure of exploration and survival.

% AI Text 2: Embark on an infinite odyssey through the ever-shifting corridors of the Backrooms, where every step brings forth a new enigma to unravel. Delve into the eerie solitude of this procedurally generated realm, where the boundaries between reality and the unknown blur into obscurity. Your quest is clear: navigate the labyrinthine passages, scavenge for crucial resources, and decipher cryptic clues to uncover the elusive escape routes scattered throughout. Will you muster the courage to confront the mysteries lurking within, or will you become lost amidst the haunting depths of the Backrooms? The choice rests with you in this gripping tale of exploration and survival.

\subsubsection{Controls}
\begin{center}
    \begin{minipage}{0.5\textwidth}
        \begin{itemize}
            \item[Mouse:] Look
            \item[Left Mouse Button:] Use Item
            \item[Right Mouse Button:] Drop Item
            \item[Left Shift:] Run
            \item[WASD Keys:] Move
            \item[F Key:] Pick Up Item
        \end{itemize}
    \end{minipage}
\end{center}

\chapter{User Manual}
% Describe how to use the Unity Editor with screenshots and any advanced mechanics not discussed in the Player Guide.


% MEETING NOTES
\chapter{Meeting Notes}
\section{Semester One Meeting Notes}
\label{sec:semester_one_meeting_notes}
\noindent Weeks 0 to 3
\begin{itemize}
    \item Research and narrowing down project aims.
    \item DOER and ethics submissions.
\end{itemize}
\noindent Week 4 - 03.10.2023
\begin{itemize}
    \item Create a design outline document.
    \item Try to adapt one of the WFC implementations to 3D.
    \begin{itemize}
        \item \url{https://github.com/GarnetKane99/WaveFunctionCollapse}
        \item Up, down, right, left, front, back neighbours.
        \item Each prefab in the neighbours array should have a weighting too. Use an entropy calculation.
    \end{itemize}
    \item Get a simple, one-room playable demo.
\end{itemize}

\noindent Week 5 - 10.10.2023
\begin{itemize}
    \item No meeting ILW.
    \item Adapt research into a draft context / literature review for the final report (see email for lit rev guide).
    \item Get WFC working with weights by the next meeting.
\end{itemize}

\noindent Week 7 - 24.10.2023
\begin{itemize}
    \item Try to finish literature and make an introduction to WFC in the report.
    \item Implement your own WFC.
\end{itemize}

\noindent Week 8 - 31.10.2023
\begin{itemize}
    \item Continue working on WFC algorithm.
\end{itemize}

\noindent Week 9 - 10.11.2023
\begin{itemize}
    \item Fix last bugs with MAC.
    \item Either get in symmetries or give each neighbour data explicitly for now to get generation working properly.
    \item Lowest entropy.
\end{itemize}

\noindent Week 10 - 17.11.2023
\begin{itemize}
    \item Implement weighting into the algorithm.
    \item Have another look at what is meant by entropy.
    \item Make notes on the directional artefacts (probably a result from overfitting / tileset neighbour bias).
\end{itemize}

\noindent Week 11 - 24.11.2023
\begin{itemize}
    \item Visual debugger to help with artefacts (diagonal, edge only shapes...)
    \item Refocus a bit on the game element (texturing, filter, ...)
    \item Extra: Infinite modifying in blocks, sub-wfc.
\end{itemize}

\section{Semester Two Meeting Notes}
\label{sec:semester_two_meeting_notes}
\noindent Week 1 - 17.01.2024
\begin{itemize}
    \item Goal block
    \item Have player start in middle of grid with empty blocks around at start of generation.
    \item Look more into the lighting. Replace placeholder carpet and roof. Check performance of tiling roof.
\end{itemize}

\noindent Week 2 - 24.01.2024
\begin{itemize}
    \item Main Goals:
    \begin{itemize}
        \item Continue working towards chunk generation.
        \begin{itemize}
            \item Let chunks that are separately generated overlap information.
            \begin{itemize}
                \item Chunks have trouble generating. Maybe the overlap method is not fully correct. Think about overlap size and simply getting information vs delete some part.
                \item Infinite modifying in blocks with good chunk sizes might help without backtracking.
            \end{itemize}
            \item Allow backtracking for chunks.
        \end{itemize}
        \item Script goal to let player win.
        \begin{itemize}
            \item Detect when the player touches the goal.
            \item Show a message and go back to the menu / teleport the player somewhere else.
        \end{itemize}
    \end{itemize}
    \item Extra:
    \begin{itemize}
        \item Add in a mini sub-region to test (i.e freezer).
        \item Think about making a variant for empty vs empty light.
        \item Graphics improvements:
        \begin{itemize}
            \item Fog / render distance if needed later.
            \begin{itemize}
                \item Adjusted fog colour.
                \item Should anything else be adjusted? For now, no render distance cust. needed.
            \end{itemize}
            \item Make vhs effect screen space rather than plane.
            \begin{itemize}
                \item Seems to be difficult to overlay onto the render texture without more cameras.
            \end{itemize}
            \item Skybox lighting for ambient light.
            \item Balance light settings and bloom a bit more.
        \end{itemize}
        \item Audio improvements:
        \begin{itemize}
            \item Script the lights to have random offset.
            \begin{itemize}
                \item Should probably also replace the buzz with something that doesn’t tick.
            \end{itemize}
            \item Add VHS SFX to the player.
        \end{itemize}
    \end{itemize}
\end{itemize}

\noindent Week 3 - 31.01.2024
\begin{itemize}
    \item Only discussed progress per email.
    \item Implement infinite modifying in blocks for next week.
\end{itemize}

\noindent Week 4 - 07.02.2024
\begin{itemize}
    \item Do report writing.
    \item Polish game and finish any 90\% implemented features.
    \item ((Test performance of 2D grid vs linear grid. Might be changed automatically by C\#. Probably done automatically))
\end{itemize}

\noindent Week 5 - 15.02.2024
\begin{itemize}
    \item Focussing more on other projects at the moment.
    \item Tried to do parallelisation, but it wasn’t simple. Focus on other things first.
    \begin{itemize}
        \item Generate the level as the player moves about.
    \end{itemize}
    \item Continue report writing when able.
\end{itemize}

\noindent Week 6 - 22.02.2024
\begin{itemize}
    \item Very busy with other courseworks.
    \item Continue the report in chunks over the break.
\end{itemize}


% DESIGN DOCUMENT
\chapter{Design Document}
\section*{Outline}
\begin{itemize}
    \item Backrooms-style game using Wave Function Collapse (WFC) to infinitely generate rooms.
    \item Different tilesets can be used to generate different levels.
\end{itemize}

\section*{Generator}
\begin{itemize}
    \item First generate a fixed size level of a few tiles. Later, generate in chunks (modifying in infinite blocks) to support infinite generation as well as with more tiles.
    \item Generate geometry for each level. Give “doors” a weighting to go into another sub-level.
    \item Is it worth integrating some more traditional methods? Traditional generators may work better for rooms. Something more random like WFC might work better for Backrooms.
\end{itemize}

\section*{Modules / Blocks}
\begin{itemize}
    \item Each a 1x1 tile, Empyrion style block with adjacency constraints and weighting (how?) for each.
    \begin{itemize}
        \item Full
        \begin{itemize}
            \item Floor (full with carpet on top)
            \item Ceiling (full with ceiling on bottom)
            \item Could do all three in one block type? Or maybe use texture variations somehow.
        \end{itemize}
        \item Wall (half block)
        \begin{itemize}
            \item Crouch hole (rotated wall)
        \end{itemize}
        \item Corner (half)
        \item Corner (quarter)
        \item Thin Wall (quarter)
        \item Corner pillar (half)
        \item Corner pillar (quarter)
        \item Ceiling tiles
        \begin{itemize}
            \item A single tile consisting of multiple tiles (2x2?). Above empty, give each sub-tile a chance of of being emissive.
        \end{itemize}
        \item Empty
        \item Centre pillar
        \item Miscellaneous half block variants that fit other normal blocks for variety.
    \end{itemize}
\end{itemize}

\section*{Regions / Structure}
\begin{itemize}
    \item Make a ``door'' tile that is a floor tile with adjacencies that connect several sublevels.
    \item Is there a better way of doing this that maintains modularity?
    \item How might room structures be made while maintaining modularity?
    \begin{itemize}
        \item Not needed for some areas of the backrooms (i.e. starting area) as they are not well structured.
    \end{itemize}
    \item Use a density noise map that defines how empty or full an area of the map is. Helps give natural variety.
    \begin{itemize}
        \item Can also apply noise to lights and grunge for textures.
    \end{itemize}
\end{itemize}

\section*{Goal}
\begin{itemize}
    \item Collect key items across sublevels (and levels?) to unlock new areas / objectives (Monstrum escape route style).
\end{itemize}

\chapter{Holiday and Semester Two Task List}\label{sec:tasks}
\begin{itemize}
    \item Algorithm
    \begin{itemize}
        \item Make the algorithm visual (intermediate steps)
        \item Give control of generation to the player via keyboard when debugging.
        \item Infinite modifying in blocks
        \begin{itemize}
            \item Think about how this will affect modularity. Will implementing this mean changes to the base algorithm will be harder to make? Or will this just always work on top? It will probably work separately.
        \end{itemize}
        \item Sub-WFC
        \item Generate block variants for symmetries from the initial block.
    \end{itemize}
    \item Audio
    \begin{itemize}
        \item Get ambience sounds
        \item Footstep sounds
        \item Audio filter
    \end{itemize}
    \item Game
    \begin{itemize}
        \item Objectives
        \begin{itemize}
            \item Spawn items for objectives
            \item Spawn key locations among random generation
        \end{itemize}
        \item Set up player in Unity to start in the middle of the maze
        \item Make a menu screen to start the game.
    \end{itemize}
    \item Graphics
    \begin{itemize}
        \item Textures for each block
        \begin{itemize}
            \item Got a low res wallpaper texture and simple colour for rims. Should help performance.
            \item In the future, use AI art for some textures? (Just a possibility)
        \end{itemize}
        \item Filters for the camera
        \begin{itemize}
            \item First round of basic effects. Could probably be enhanced in quality later.
        \end{itemize}
        \item Graphics for floor and lights.
        \begin{itemize}
            \item Have a flat texture for the whole map? Or break it into block chunks too? Performance?
            \item Get proper lighting in that is also performant.
        \end{itemize}
    \end{itemize}
    \item Report
    \begin{itemize}
        \item Break into sections.
        \begin{itemize}
            \item Game design, algorithms, discuss different options and things tried, audio and graphics...
        \end{itemize}
    \end{itemize}
\end{itemize}