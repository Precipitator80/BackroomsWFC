% Language setting
\usepackage[UKenglish]{babel}
\usepackage[utf8]{inputenc}
\usepackage{csquotes}
\usepackage{graphicx} % Required for inserting images
% Set page size and margins
% Replace `letterpaper' with`a4paper' for UK/EU standard size
\usepackage[a4paper,top=2.5cm,bottom=2.5cm,left=3cm,right=3cm,marginparwidth=1.75cm]{geometry}

\usepackage[pdftex,
    pdfauthor={\Author},
    pdftitle={\Title},
    pdfsubject={\Subject},
    pdfkeywords={\Keywords},
    colorlinks=true,
    allcolors=blue]{hyperref}
% ------------------------------ %
% Various useful packages
\usepackage{amsmath}
\usepackage{graphicx}
\usepackage{multirow}
\usepackage{float}
% IEEE bibliography setup.
\usepackage[backend=biber, style=ieee, isbn=false,sortcites, maxbibnames=6, minbibnames=1]{biblatex}
\addbibresource{ref.bib} % The references.bib file in which the bibliography used should be.
\usepackage{tabularx}
\usepackage{amssymb}
\usepackage{colortbl}
\usepackage{tikz}
\usepackage[T1]{fontenc} % Extra font options such as small capital (textsc).
\usepackage{lipsum} % Lorem Ipsum Package for placeholder text.
%\usepackage{comment} % Multi-line / Block comments. Doesn't seem to work.

% Enumerate tag using the alphabet instead of numbers - A.Ellett - https://tex.stackexchange.com/questions/129951/enumerate-tag-using-the-alphabet-instead-of-numbers - Accessed 15.04.2023
\usepackage{enumitem} % Better list customisation

% Frames for enumerate and itemize.
\usepackage{framed}
% ------------------------------ %
% Subfigures
\usepackage{subfigure}
\usepackage{subcaption}
% ------------------------------ %
% Node Graphs
\usepackage{tikz}
\usetikzlibrary{positioning}
%\begin{tikzpicture}[
%roundnode/.style={circle, draw=green!60, fill=green!5, very thick, minimum size=7mm},
%squarednode/.style={rectangle, draw=red!60, fill=red!5, very thick, minimum size=5mm},
%]
%%Nodes
%\node[squarednode]      (maintopic)                              {2};
%\node[roundnode]        (uppercircle)       [above=of maintopic] {1};
%\node[squarednode]      (rightsquare)       [right=of maintopic] {3};
%\node[roundnode]        (lowercircle)       [below=of maintopic] {4};
%
%%Lines
%\draw[->] (uppercircle.south) -- (maintopic.north);
%\draw[->] (maintopic.east) -- (rightsquare.west);
%\draw[->] (rightsquare.south) .. controls +(down:7mm) and +(right:7mm) .. (lowercircle.east);
%\end{tikzpicture}
% ------------------------------ %
% Acronyms / Glossaries
\usepackage[acronym]{glossaries}
\renewcommand*{\glstextformat}[1]{\textcolor{black}{#1}}
% VS Code - LaTeX Workshop usage of glossaries package - Shady Puck - https://tex.stackexchange.com/questions/536519/vs-code-latex-workshop-usage-of-glossaries-package - Accessed 19.03.2024
% How do I remove color from abbreviations? - Tim Hilt - https://tex.stackexchange.com/questions/532272/how-do-i-remove-color-from-abbreviations - Accessed 19.03.2024
% ------------------------------ %
% Allow urls to break more cleanly in the bibliography.
\setcounter{biburllcpenalty}{100}
\setcounter{biburlucpenalty}{100}
\setcounter{biburlnumpenalty}{100}
%\emergencystretch=1em % Might help with other spills.

% Latex Bibliography Reference Going Off The Page - James - https://stackoverflow.com/questions/43590245/latex-bibliography-reference-going-off-the-page - Accessed 13.02.2024
% BibLaTeX long URL extending into border despite using biburllcpenalty - moewe - https://tex.stackexchange.com/questions/466114/biblatex-long-url-extending-into-border-despite-using-biburllcpenalty - Accessed 13.02.2024
% ------------------------------ %
% Code listing - Overleaf - https://www.overleaf.com/learn/latex/Code_listing - Accessed 17.10.2023
% How to make a figure with code? - James - https://tex.stackexchange.com/questions/503533/how-to-make-a-figure-with-code - Accessed 17.10.2023
\usepackage{listings}
\usepackage{xcolor}

\definecolor{codegreen}{rgb}{0,0.6,0}
\definecolor{codegray}{rgb}{0.5,0.5,0.5}
\definecolor{codepurple}{rgb}{0.58,0,0.82}
\definecolor{backcolour}{rgb}{0.95,0.95,0.95}

\lstdefinestyle{mystyle}{
    backgroundcolor=\color{backcolour},
    commentstyle=\color{codegreen},
    keywordstyle=\color{magenta},
    numberstyle=\tiny\color{codegray},
    stringstyle=\color{codepurple},
    basicstyle=\ttfamily\scriptsize, % Font size (i.e. \scriptsize or \footnotesize)
    breakatwhitespace=false,
    breaklines=true,
    captionpos=b,
    keepspaces=true,
    numbers=left,
    numbersep=5pt,
    showspaces=false,
    showstringspaces=false,
    showtabs=false,
    tabsize=2,
    aboveskip=20pt,
}

\lstset{style=mystyle}

%\noindent\begin{minipage}{\linewidth}
%\begin{lstlisting}[caption={captiontext}, label={code:labelname}, frame=single]
%\end{lstlisting}
%\end{minipage}
% ------------------------------ %
% Headers
\usepackage{fancyhdr}
\pagestyle{fancy}
%\lhead{\Title}
%\rhead{\Author}
\setlength{\headheight}{15pt}

% Report
%\renewcommand{\chaptermark}[1]{\markboth{#1}{}}
%\renewcommand{\sectionmark}[1]{\markright{#1}}
%\lhead{\nouppercase{\chaptername\ \thechapter. \leftmark}}
%\rhead{\nouppercase{\thesection. \rightmark}}

\lhead{\nouppercase\leftmark}
\rhead{\nouppercase\rightmark}

% Article - Or instead just use the commented out code at the top?
%\renewcommand{\sectionmark}[1]{\markright{#1}}
%\lhead{\nouppercase{\sectionname\ \thesection. \leftmark}}
%\rhead{\nouppercase{\rightmark}}
% ------------------------------ %
% Custom commands
% Guide text: Substitution
\newcommand{\substitution}[1]{[\textit{#1}]}

% Guide text: Optional
\newcommand{\optional}[1]{[#1]}

% Full Reference to Section
\newcommand*{\fullref}[1]{\hyperref[{#1}]{\autoref*{#1} \nameref*{#1}}} % One single link
% ------------------------------ %
% Other References:
% set a maximum width and height for an image - Yiannis Lazarides - https://tex.stackexchange.com/questions/47245/set-a-maximum-width-and-height-for-an-image - Accessed 16.04.2023
%\begin{figure}[H]
%\centering
%\includegraphics[width=\textwidth, height=0.3\textheight, keepaspectratio]{ImagePath}
%\caption{captionText}
%\label{fig:figureName}
%\end{figure}
% ------------------------------ %
% Scrap code
%\usepackage{geometry}
%\geometry{a4paper, total={170mm,257mm}, left=5mm, top=5mm}

% OLD WAY OF HAVING TITLE VARIABLES
% Edit these for each new project. Make sure to use "\makeatletter" in each new tex file.
% Use the values of \title, \author and \date on a custom title page - Martin Scharrer - https://tex.stackexchange.com/questions/10130/use-the-values-of-title-author-and-date-on-a-custom-title-page - Accessed 12.09.2023